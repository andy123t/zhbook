% !TEX program = xelatex
% 使用 texlive 完整编译:
% xelatex -> bibtex -> xelatex -> xelatex
% zhbook 中文书籍 LaTeX 模板

%--- 正文前后都没有空白页 ---
\documentclass[openany,twoside,zihao=-4]{zhbook}
% print 用于打印, 封面等生成空白页

%--- 正文前后都有空白页, 正文一章结束可空白页使新的一章是在奇数页开始 ---
%\documentclass[twoside,openright,zihao=-4]{zhbook}


% 进行信息设置
\title{中文书籍 \LaTeX{} 模板  \\[10pt] \zihao{2} 这是书籍模板的副标题}
\college{某某大学}
\author{某某某}  % 作者姓名
%\date{2\,0\,2\,X~年~X~月}  % 日期
\bioinfo{其他信息}
\pubyear{202X}  % 出版年
\publisher{出版社}

% 通用虚拟出版商标志
\newcommand{\plogo}{\fbox{$\mathcal{PL}$}}


%----- 添加其他宏包 -----
\usepackage{listings}
\usepackage{subfig}
%\usepackage{xltxtra}

%----- 取消链接颜色和方框 -----
%\hypersetup{hidelinks}

%----- 参考文献格式 -----
%\bibliographystyle{plain} % abbrv, unsrt, siam
\bibliographystyle{thuthesis-numeric}
%\bibliographystyle{thuthesis-author-year}

%----- 参考文献引用格式 -----
\usepackage[numbers,sort&compress]{natbib}
%\usepackage[numbers,super,square,sort&compress]{natbib}
%\usepackage[authoryear,sort&compress]{natbib}
\def\bibfont{\small}      % 修改参考文献字体
\setlength{\bibsep}{8pt}  % 调整参考文献间距

%----- 制作索引 -----
\usepackage{makeidx}
\makeindex

%----- 调整列表项的间距 -----
%\setlength{\itemsep}{3pt plus 1pt minus 1pt}

%----- 设置英文字体 -----
\usepackage{newtxtext}
%\setmainfont{Times New Roman}

%----- 设置数学字体 -----
%\usepackage{newtxmath}
%\usepackage{mathptmx}

%----- 定义符号描述命令  -----
\newcommand{\nameditem}[3][]{
\noindent\hspace{2em}\makebox[0.2\textwidth][l]{#2}{{#3}
\hfill\makebox[0.2\textwidth][l]{#1}\hspace*{2em}}\par}

%----- 插入 PDF 文件命令 -----
%\includepdf[pages=-]{pdfname.pdf}

%----- 微分算子 -----
\newcommand*{\dif}{\mathop{}\!\mathrm{d}}

%----- 自定义命令 -----
\newcommand{\CC}{\ensuremath{\mathbb{C}}}
\newcommand{\RR}{\ensuremath{\mathbb{R}}}
\newcommand{\A}{\mathcal{A}}
\newcommand{\bA}{\boldsymbol{A}}
\newcommand{\ii}{\mathrm{i}\,}
\newcommand{\abs}[1]{\lvert#1\rvert}
\newcommand{\norm}[1]{\left\lVert#1\right\rVert}
\newcommand{\dx}[1][x]{\mathop{}\!\mathrm{d}#1}
\newcommand{\red}[1]{\textcolor{red}{#1}}


\begin{document}


% 生成封面
\maketitle

% 插入封面 PDF文件
%\thispagestyle{empty}
%\includepdf{cover.pdf}
%\cleardoublepage


%%%%%%%%%%%%%%%%%%%%%%%%%%%%%%%%%%%%%%%%%%%%%%%%%%%

\frontmatter
%\pagenumbering{Roman} % 摘要页码为大写罗马数字

%%%%%%%%%%%%%%%%%%%%%% 前言 %%%%%%%%%%%%%%%%%%%%%%%%


%%%%%%%%%%%%%%%%%%%% 前言 %%%%%%%%%%%%%%%%%%%%%

\begin{preface}

前言内容前言内容前言内容前言内容前言内容前言内容前言内容前言内容前言内容前言内容前言内容前言内容前言内容前言内容前言内容前言内容前言内容前言内容前言内容前言内容前言内容前言内容前言内容前言内容前言内容前言内容前言内容前言内容前言内容前言内容前言内容前言内容前言内容前言内容前言内容前言内容前言内容前言内容前言内容前言内容前言内容前言内容前言内容前言内容前言内容.

前言内容前言内容前言内容前言内容前言内容前言内容前言内容前言内容前言内容前言内容前言内容前言内容前言内容前言内容前言内容前言内容前言内容前言内容前言内容前言内容前言内容前言内容前言内容前言内容前言内容前言内容前言内容前言内容前言内容前言内容前言内容前言内容前言内容前言内容前言内容前言内容前言内容前言内容前言内容前言内容前言内容前言内容前言内容前言内容前言内容.


\end{preface} 


%%%%%%%%%%%%%%%%% 中文摘要内容和关键字  %%%%%%%%%%%%%%

\input{part/cnabstract}


%%%%%%%%%%%%%%%%% 英文摘要内容和关键字 %%%%%%%%%%%%%%

\input{part/enabstract}


%%%%%%%%%%%%%%%%%%%%%%% 目录 %%%%%%%%%%%%%%%%%%%%%%%

% 生成目录
\maketoc

% 生成插图清单, 如不需要可以注释
\makelof

% 生成表格清单, 如不需要可以注释
\makelot

%%%%%%%%%%%%%%%%%%%% 主要符号表 %%%%%%%%%%%%%%%%%%%%%

% 如不需要可以注释
\input{part/denotation}


%%%%%%%%%%%%%%%%%%%%%%%%%%%%%%%%%%%%%%%%%%%%%%%%%%%%

\mainmatter

%%%%%%%%%%%%%%%%%%%% 正文内容 %%%%%%%%%%%%%%%%%%%%%%%


%----- 第1章 引言 -----

%%%%%%%%%%%%%%%%%%%% 引言 %%%%%%%%%%%%%%%%%%%%


\chapter{引言}\label{chap:Intro}

\section{研究背景}\label{sec:background}

这是小四号的正文字体, 行间距 1.35 倍.

通过空一行实现段落换行, 仅仅是回车并不会产生新的段落.

自定义了一个命令 \verb|\red{文字}| 可以\red{加红文字}, 可以在论文修改阶段方便标记.

这是一个文献引用的示例 \cite{Tadmor2012} 和 \cite{LiLiu1997,Adams2003,TreWei2014}.

\section{系统要求}
本模板 \href{https://github.com/andy123t/zhbook}{\texttt{zhbook}} 基于标准文类 ctexbook 设计, 所以 ctexbook 文类的选项也能传递给本模板, 比如 \lstinline{openany, twoside} 等. 本模板可以在目前主流的 \href{https://en.wikibooks.org/wiki/LaTeX/Introduction}{\LaTeX{}} 编译系统中使用, 如 \TeX{}Live 和 MiK\TeX{}. 因 C\TeX{} 套装已停止维护, \textbf{不再建议使用}.

这是一大段文字这是一大段文字这是一大段文字这是一大段文字这是一大段文字这是一大段文字这是一大段文字这是一大段文字这是一大段文字这是一大段文字这是一大段文字这是一大段文字这是一大段文字这是一大段文字


\section{结构安排}

本文接下来的写作安排如下:

第二章, 我们介绍了 LaTeX 常用环境, 包括列表的使用、文献引用、数学公式、定理环境以及算法环境.

第三章, 对于差分方法数值求解微分方程, 给出了一个简短的示例.

第四章, 针对插图环境, 给出了单个图形居中放置、两个图形并排放置以及多个图形并排放置的示例.

第五章, 针对表格环境, 介绍了一些自定义命令, 并给出相应的表格插入示例.

最后是参考文献、附录和后记.




%----- 第2章 LaTeX 常用环境 -----
\input{part/chap02}

%----- 第3章 微分方程的数值方法 -----
\input{part/chap03}

%----- 第4章 插图环境 -----
\input{part/chap04}

%----- 第5章 表格环境 -----
\input{part/chap05}


%%%%%%%%%%%%%%%%%%%%%% 参考文献 %%%%%%%%%%%%%%%%%%%%%

% 生成参考文献, 两种方式任选一种

% 第一种方式, 使用 bib 文件
%\nocite{*}  % 可以显示全部参考文献
\bibliography{reference}

%--------------------------------------------------%

% 第二种方式, 手动添加文献信息
%\input{part/bibliography}


%%%%%%%%%%%%%%%%%%%%%% 附录 %%%%%%%%%%%%%%%%%%%%%%%%

% 添加附录, 如不需要可以注释
\input{part/appendix}


%%%%%%%%%%%%%%%%%%%%%%%%%%%%%%%%%%%%%%%%%%%%%%%%%%%%

\backmatter  % 结束章节自动编号

%%%%%%%%%%%%%%%%%%%%%% 索引  %%%%%%%%%%%%%%%%%%%%%%%%

\clearpage
\renewcommand\indexname{索~~引}
\phantomsection
\addcontentsline{toc}{chapter}{索~~引}
\printindex


%%%%%%%%%%%%%%%%%%%%%%% 后记 %%%%%%%%%%%%%%%%%%%%%%%%


%%%%%%%%%%%%%%%%%%%% 后记 %%%%%%%%%%%%%%%%%%%%%

\chapter{后~~记}

后记内容后记内容后记内容后记内容后记内容后记内容后记内容后记内容后记内容后记内容后记内容后记内容后记内容后记内容后记内容后记内容后记内容后记内容后记内容后记内容后记内容后记内容后记内容后记内容后记内容后记内容后记内容后记内容后记内容后记内容后记内容后记内容后记内容后记内容后记内容后记内容后记内容后记内容后记内容后记内容后记内容后记内容后记内容后记内容后记内容后记内容后记内容后记内容后记内容后记内容后记内容后记内容后记内容后记内容后记内容后记内容后记内容后记内容后记内容后记内容后记内容后记内容后记内容后记内容后记内容后记内容后记内容后记内容后记内容


\vspace{5ex}
\begin{flushright}
作~~者~~~~~~~~~

202X年X月~~~~~
\end{flushright}




\end{document}


