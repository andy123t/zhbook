
%%%%%%%%%%%%%%%%%%%% 引言 %%%%%%%%%%%%%%%%%%%%%

\chapter{引言}\label{chap:Intro}

\section{研究背景}\label{sec:background}

这是小四号的正文字体, 行间距 1.35 倍.

通过空一行实现段落换行, 仅仅是回车并不会产生新的段落.

自定义了一个命令 \verb|\red{文字}| 可以\red{加红文字}, 可以在论文修改阶段方便标记.

这是一个文献引用的示例 \cite{Tadmor2012} 和 \cite{LiLiu1997,Adams2003,TreWei2014}.

\section{系统要求}
本模板 \href{https://github.com/andy123t/zhbook}{\texttt{zhbook}} 基于标准文类 ctexbook 设计, 所以 ctexbook 文类的选项也能传递给本模板, 比如 \lstinline{openany, twoside} 等. 本模板可以在目前主流的 \href{https://en.wikibooks.org/wiki/LaTeX/Introduction}{\LaTeX{}} 编译系统中使用, 如 \TeX{}Live 和 MiK\TeX{}. 因 C\TeX{} 套装已停止维护, \textbf{不再建议使用}.

这是一大段文字这是一大段文字这是一大段文字这是一大段文字这是一大段文字这是一大段文字这是一大段文字这是一大段文字这是一大段文字这是一大段文字这是一大段文字这是一大段文字这是一大段文字这是一大段文字


\section{结构安排}

本文接下来的写作安排如下:

第二章, 我们介绍了 LaTeX 常用环境, 包括列表的使用、文献引用、数学公式、定理环境以及算法环境.

第三章, 对于差分方法数值求解微分方程, 给出了一个简短的示例.

第四章, 针对插图环境, 给出了单个图形居中放置、两个图形并排放置以及多个图形并排放置的示例.

第五章, 针对表格环境, 介绍了一些自定义命令, 并给出相应的表格插入示例.

最后是参考文献、附录和后记.

