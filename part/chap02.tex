
%%%%%%%%%%%%%%%%%%%% LaTeX 常用环境 %%%%%%%%%%%%%%%%%%%%

\chapter{LaTeX 常用环境}

\section{列表的使用}

这是一个计数的列表.
\begin{enumerate}%[label={(\arabic*)}]
	\item 第一项
		\begin{enumerate}
			\item 第一项中的第一项
			\item 第一项中的第二项
		\end{enumerate}
	\item 第二项
	\item 第三项
\end{enumerate}

这是一个不计数的列表.
\begin{itemize}%[label={$\bullet$}]
	\item 第一项
	\begin{itemize}
		\item 第一项中的第一项
		\item 第一项中的第二项
	\end{itemize}
	\item 第二项
	\item 第三项
\end{itemize}


\section{文献引用}

参考文献可采用 BibTeX 的方式生成 (内容写在文件 \verb|mybib.bib| 中), 参考文献的样式为 \verb|thuthesis-numeric| (对应的引用格式可选 \verb|numbers| 或  \verb|super|)和 \verb|thuthesis-author-year| (对应的引用格式 \verb|authoryear|), 符合国家标准《信息与文献参考文献著录规则》GB/T 7714-2015, 论文中引用和参考的文献必须列出. 参考文献序号按所引文献在论文中出现的先后次序排列. 引用文献应在论文中的引用处加注文献序号, 并加注方括弧.

文献引用示例 \cite{LiLiu1997} 和 \cite{Adams2003,Shen1994}.


\section{数学公式}\label{sec:mathEqEnv}

数学公式的使用请参考公式手册 symbols-a4, 或者 《一份(不太)简短的 \LaTeX~2$\varepsilon$ 介绍》 (lshort-zh-cn).

自定义命令表示的几个数学符号 $\RR$, $\CC$, $\A$, $\ii$, $\md$, $\bA$.

在文中行内公式可以这么写: $a^2+b^2=c^2$, 这是勾股定理, 它还可以表示为 $c=\sqrt{a^2+b^2}$, 还可以让公式单独一段并且加上编号
\begin{equation}\label{eq:trifun}
\sin^2{\theta}+\cos^2{\theta}=1.
\end{equation}
还可以通过添加标签在正文中引用公式, 如等式~\eqref{eq:trifun} 或者 \ref{eq:trifun}.

读者可能阅读过其它手册或者资料, 知道 LaTeX 提供了 eqnarray 环境. 它按照等号左边—等号—等号右边呈三列对齐, 但等号周围的空隙过大, 加上公式编号等一些 bug, 目前已不推荐使用. (摘自 lshort-zh-cn)

多行公式常用 align 环境, 公式通过 \verb|&| 对齐. 分隔符通常放在等号左边:
\begin{align}
a & = b + c \\
& = d + e.
\end{align}

align 环境会给每行公式都编号. 我们仍然可以用 \verb|\notag| 或 \verb|\nonumber| 去掉某行的编号. 在以下的例子,
为了对齐等号, 我们将分隔符放在右侧, 并且此时需要在等号后添加一对括号 \verb|{}| 以产生正常的间距:
\begin{align}
a ={} & b + c \\
={} & d + e + f + g + h + i + j \notag \\
& + m + n + o \\
={} & p + q + r + s.
\end{align}

如果不需要按等号对齐, 只需罗列数个公式, gather 将是一个很好用的环境:
\begin{gather}
a = b + c \\
d = e + f + g \notag \\
h + i = j
\end{gather}

align 和 gather 有对应的不带编号的版本 align* 和 gather*.
对于 align、 gather、align* 与 gather* 等环境, 若添加命令 \verb|\allowdisplaybreaks| 后 (已添加), 公式可以跨页显示.

多个公式组在一起公用一个编号, 编号位于公式的居中位置, amsmath 宏包提供了诸如 aligned、gathered 等环境, 与 equation 环境套用.

这个公式使用 aligned 环境 (\textbf{推荐使用})
\begin{equation}\label{eq:alignedEq}
\left\{\begin{aligned}
  &-\frac{\mathrm{d}^{2} u}{\mathrm{d} x^{2}}+\frac{\mathrm{d} u}{\mathrm{d} x}=\pi^{2} \sin (\pi x)+\pi \cos (\pi x),\quad x \in [0,1], \\
  &u(0)=0,\quad u(1)=0.
\end{aligned} \right.
\end{equation}
其中方程的解析解为 $u=\sin(\pi x)$.

这个公式使用 array 环境
\begin{equation}\label{eq:arrayEq}
\left\{\begin{array}{l}
\displaystyle
-\frac{\mathrm{d}^{2} u}{\mathrm{d} x^{2}}+\frac{\mathrm{d} u}{\mathrm{d} x}=\pi^{2} \sin (\pi x)+\pi \cos (\pi x),\quad x \in [0,1], \\[6pt]
u(0)=0,\quad u(1)=0.
\end{array} \right.
\end{equation}

aligned 与 equation 环境套用, 公式间距自动调节, 如果有分式, 分式也是行间显示. 如果用 array 与 equation 环境套用, 需要手动调整公式行间距和行间显示.


\section{定理环境}

\begin{definition}\label{def:foo}
这是一个定义.
\end{definition}

\begin{proposition}\label{prop:foo}
这是一个命题.
\end{proposition}

\begin{lemma}[Lemma]\label{lmm:foo}
这是一个引理.
\end{lemma}

\begin{theorem}[Theorem]\label{thm:foo}
这是一个定理.
\end{theorem}
\begin{proof}
这是证明环境.
\end{proof}

\begin{corollary}\label{cor:foo}
这是一个推论.
\end{corollary}

\begin{proposition}[Proposition]
这是一个命题.
\end{proposition}

\begin{lemma}\label{lmm:convergence} {\rm (\textit{参考文献}\cite{LiLiu1997})}
假设单步法具有 $p$ 阶精度, 且増量函数 $\varphi(x_{n}, u_{n}, h)$ 关于 $u$ 满足 \textup{Lipschitz} 条件
\begin{equation}\label{eq:conver1}
|\varphi(x, u, h)-\varphi(x, \bar{u}, h)| \leqslant L_{\varphi}|u-\bar{u}|.
\end{equation}
\end{lemma}

\begin{theorem}\label{thm:convergence}
假设单步法具有 $p$ 阶精度, 且増量函数 $\varphi(x_{n}, u_{n}, h)$ 关于 $u$ 满足\textup{Lipschitz} 条件
\begin{equation}\label{eq:conver2}
|\varphi(x, u, h)-\varphi(x, \bar{u}, h)| \leqslant L_{\varphi}|u-\bar{u}|.
\end{equation}
\end{theorem}
\begin{proof}[\normalfont\bfseries 证明~\nopunct]
由定理 \ref{lmm:convergence} 和 \eqref{eq:alignedEq} 式可以推出以上结论.
\end{proof}

\begin{corollary}\label{col:convergence}
假设单步法具有 $p$ 阶精度, 且増量函数 $\varphi(x_{n}, u_{n}, h)$ 关于 $u$ 满足 \textup{Lipschitz} 条件
\begin{equation}\label{eq:conver3}
|\varphi(x, u, h)-\varphi(x, \bar{u}, h)| \leqslant L_{\varphi}|u-\bar{u}|.
\end{equation}
\end{corollary}

\begin{remark}\label{rem:remark}
这是一个 remark.
\end{remark}

\begin{example}
这是一个例子.
\end{example}


\clearpage
\section{算法环境}

如下是算法~\ref{alg:euclid}.
\begin{algorithm}[H]
\small
\caption{~Euclid's algorithm}\label{alg:euclid}
\begin{algorithmic}[1]
  \Procedure{Euclid}{$a,b$}\Comment{The g.c.d. of a and b}
  \State $r\gets a\bmod b$
  \While{$r\not=0$}\Comment{We have the answer if r is 0}
  \State $a\gets b$
  \State $b\gets r$
  \State $r\gets a\bmod b$
  \EndWhile\label{euclidendwhile}
  \State \Return $b$\Comment{The gcd is b}
  \EndProcedure
\end{algorithmic}
\end{algorithm}


如下是算法~\ref{alg:foo}, 算法宽度可以通过 minipage 宏包调节.
\begin{center}
\vspace{-2ex}
\begin{minipage}{.9\linewidth}
\begin{algorithm}[H]
\caption{~算法的名字}\label{alg:foo}
\begin{algorithmic}[1]
\Require input parameters A, B, C
\Ensure output result
\State some description 算法介绍
\For{condition}
  \State ...
  \If{condition}
    \State ...
    \Else
    \State ...
  \EndIf
\EndFor
\While{condition}
  \State ...
\EndWhile
\State \Return result
\end{algorithmic}
\end{algorithm}
\end{minipage}
\end{center}


\section{\texorpdfstring{{\boldmath$Z=X \cup Y$}}{Z = X union Y} 的情况}

这是一个小节标题中出现数学符号的情况.

