
%%%%%%%%%%%%%%% 微分方程的数值方法 %%%%%%%%%%%%%

\chapter{微分方程的数值方法}

本章我们考虑具有以下微分方程:
\begin{equation}\label{eq:PDE}
\left\{\begin{aligned}
& L u=-\frac{{\dif}^{2} u}{\dif x^{2}}+\frac{\dif u}{\dif x}+q u=f, \quad a < x < b, \\
& u(a)=\alpha, \quad u(b)=\beta.
\end{aligned}\right.
\end{equation}
其中 $q, f$ 为 $[a,b]$ 上的连续函数, $q \geqslant 0$; $\alpha, \beta$ 为给定常数. 这是最简单的椭圆方程第一边值问题 .

问题 \eqref{eq:PDE} 存在唯一解 (引用示例参考文献 \cite{LiLiu1997}).


\section{有限差分方法}
在偏微分方程的数值解法中, 有限差分法数学概念直观, 推导自然, 是发展较早且比较成熟的数值方法. 由于计算机只能存储有限个数据和做有限次运算, 所以任何一种用计算机解题的方法, 都必须把连续问题 (微分方程的边值问題、初值问题等) 离散化, 最终化成有限形式的线性代数方程组.

\subsection{数值格式}
将区间 $[a,b]$ 分成 $N$ 等分, 分点为
\begin{equation*}
  x_{i}=a+i h \quad i=0,1, \cdots, N,
\end{equation*}
其中 $h=(b-a) / N$. 于是我们得到区间 $I=[a,b]$ 的一个网格剖分. $x_i$ 称为网格的节点, $h$ 称为步长.

数值格式:
\begin{equation*}
  L_{h} u_{i}=-\frac{u_{i+1}-2 u_{i}+u_{i-1}}{h^{2}}+\frac{u_{i+1}-u_{i-1}}{h}+q_{i} u_{i}=f_{i},\quad 1 \leqslant j \leqslant N-1.
\end{equation*}
其中  $q_{i}=q(x_{i})$, $f_{i}=f(x_{i})$.

以上差分方程对于 $i=1,2, \cdots, N-1$ 都成立, 加上边值条件 $u_{0}=\alpha, u_{N}=\beta$, 就得到关于 $u_i$ 的差分格式:
\begin{equation}\label{eq:fdm}
\left\{\begin{aligned}
& L_{h} u_{i}=-\frac{u_{i+1}-2 u_{i}+u_{i-1}}{h^{2}}+\frac{u_{i+1}-u_{i-1}}{2h}+q_{i} u_{i}=f_{i}, ~ i=1, \cdots, N-1, \\
& u_{0}=\alpha, \quad u_{N}=\beta.
\end{aligned}\right.
\end{equation}

它的解 $u_i$ 是 $u(x)$ 在 $x=x_i$ 处的差分解.


\subsection{矩阵形式}

先定义向量 $\boldsymbol{u}$:
\begin{equation*}
  \boldsymbol{u}=(u_{1}, u_{2}, \cdots, u_{N-1})^{\mathrm{T}}.
\end{equation*}

差分格式可以写为矩阵形式:
\begin{equation*}
  \boldsymbol{A}\boldsymbol{u}=\boldsymbol{f}.
\end{equation*}
其中矩阵 $\boldsymbol{A}$、向量 $\boldsymbol{f}$ 的定义如下, 注意向量 $\boldsymbol{f}$ 的首尾元素已包含了 $x=a$ 和 $x=b$ 处的边界条件.
\begin{equation}\label{eq:matrix1}
\boldsymbol{A}=\begin{bmatrix}
\frac{2}{h^{2}}+q_{1} & \frac{1}{2h}-\frac{1}{h^{2}} &   &  &  \\[8pt]
 -\frac{1}{2h}-\frac{1}{h^{2}} & \frac{2}{h^{2}}+q_{2} & \frac{1}{2h}-\frac{1}{h^{2}}  & &  \\[8pt]
  &  &  &  &    \\
  &  \ddots  & \ddots  &  \ddots  &  \\[8pt]
  &  &  &  &    \\
  &   & -\frac{1}{2h}-\frac{1}{h^{2}} & \frac{2}{h^{2}}+q_{N-2}& \frac{1}{2h}-\frac{1}{h^{2}} \\[8pt]
  &  &  & -\frac{1}{2h}-\frac{1}{h^{2}} & \frac{2}{h^{2}}+q_{N-1}
\end{bmatrix}.
\end{equation}

上一个矩阵用了 \verb|bmatrix| 环境, 也可以使用 \verb|array| 环境.
\begin{equation}\label{eq:matrix2}
\boldsymbol{A}=\left[\begin{array}{cccccc}
\frac{2}{h^{2}}+q_{1} & \frac{1}{2h}-\frac{1}{h^{2}} &  &  &  \\[8pt]
 -\frac{1}{2h}-\frac{1}{h^{2}} & \frac{2}{h^{2}}+q_{2} & \frac{1}{2h}-\frac{1}{h^{2}}  & &  \\[8pt]
  &  &  &  &   \\
  &  \ddots  & \ddots & \ddots &  \\[8pt]
  &  &  &  &   \\
  &   & -\frac{1}{2h}-\frac{1}{h^{2}} & \frac{2}{h^{2}}+q_{N-2} & \frac{1}{2h}-\frac{1}{h^{2}} \\[8pt]
  &  &  & -\frac{1}{2h}-\frac{1}{h^{2}} & \frac{2}{h^{2}}+q_{N-1}
\end{array}\right].
\end{equation}


